% CONCLUSÃO--------------------------------------------------------------------

\chapter{CONSIDERAÇÕES FINAIS}
\label{chap:conclusao}

A proposta de investigação do Núcleo de Tecnologia do MTST busca contribuir para o debate sobre soberania digital e resistência ao colonialismo de dados, articulando teoria decolonial, análise crítica de infraestruturas tecnológicas e práticas tecnopolíticas. A pesquisa visa superar lacunas identificadas em estudos anteriores, como a superficialidade na discussão ética e a dependência excessiva de fontes internas do movimento, ao integrar perspectivas técnicas, políticas e epistemológicas.

Principais contribuições esperadas:

    \begin{itemize}
        \item \textbf{Articulação teórico-empírica inovadora:} Integração de conceitos como colonialidade do poder \cite{quijano2005}, capitalismo de vigilância \cite{Zuboff2019} e hacktivismo decolonial \cite{Faustino2023} à análise das atuações do Núcleo, demonstrando como ferramentas hegemônicas são ressignificadas para fins emancipatórios.
        \item \textbf{Metodologia híbrida:} Triangulação de análise documental, entrevistas semiestruturadas e observação participante, permitindo mapear não apenas a arquitetura técnica do artefato, mas também suas implicações éticas e políticas.
        \item \textbf{Crítica à neutralidade tecnológica:} Análise das contradições do uso de plataformas proprietárias, especialmente as vinculadas às GAFAM, em um movimento anticapitalista, vinculando-as à extração de dados periféricos e à dependência infraestrutural.
    \end{itemize}
    
Limitações e desafios prévios:

    \begin{itemize}
        \item \textbf{Acesso condicional:} A indisponibilidade de entrevistas diretas com membros do Núcleo poderá limitar a a análise subjetiva, compensada por documentos públicos e registros audiovisuais.
        \item \textbf{Complexidade técnica:} A descrição detalhada da implementação de algum artefato criado pelo Núcleo exigirá colaboração com desenvolvedores do MTST para acesso a fluxogramas e dados quantitativos.
    \end{itemize}

Recomendações para pesquisas futuras:
    \begin{itemize}
        \item Investigar o impacto de migrações para plataformas decentralizadas na eficácia política do MTST.
        \item Analisar comparativamente iniciativas similares em outros movimentos do Sul Global.
    \end{itemize}

A insurgência tecnopolítica do MTST ilustra que a luta por soberania digital não se reduz à negação de ferramentas hegemônicas, mas à reapropriação crítica destas. Este processo, ainda em gestação, exige diálogo permanente entre movimentos sociais, academia e desenvolvedores comprometidos com a justiça socioambiental na era digital.