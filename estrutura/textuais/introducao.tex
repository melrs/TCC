



% INTRODUÇÃO-------------------------------------------------------------------

\chapter{INTRODUÇÃO}
\label{chap:introducao}

A popularização e o aperfeiçoamento das tecnologias digitais, juntamente com a disseminação da Internet, são frequentemente reconhecidos como elementos centrais para o progresso econômico, político e social do século XXI \cite{Silveira2021}. A interação contínua com dispositivos conectados tornou-se um aspecto essencial e recorrente na vida cotidiana das pessoas \cite{silveira-demcodigos}, e é nesse contexto que as plataformas digitais se consolidam como ambientes de intermediação, mediando cada vez mais aspectos da vida social e reconfigurando as atividades humanas, inserindo práticas cotidianas no contexto digital, um ecossistema profundamente conectado e orientado a dados. No entanto, esse processo não ocorre de forma neutra. 

As Tecnologias da Informação e Comunicação (TICs), ao serem apropriadas pela racionalidade neoliberal, tornam-se ferramentas que dominam a linguagem e moldam o ambiente digital de modo a favorecer interesses corporativos, transformando espaços diversos em organizações focadas no lucro, reproduzindo uma sociedade incivil, em que interesses privados se sobrepõem aos públicos, com tecnologias operadas por algoritmos que reduzem a diversidade das relações \cite{souza_sabbag_achilles_2024}. Por sua vez, esses algoritmos, essência constituinte das chamadas plataformas digitais, não têm intenção própria, isto é, eles operam por causas previamente explicitadas e não por fins conscientes. Mesmo que pareçam “inteligentes” ou “autônomos”, eles seguem regras e reações determinadas por sua programação inicial e infraestrutura física \cite{Faustino2023}. As plataformas digitais, como tecnologias, estão sujeitas a abarcarem propósitos para além de seu uso imediato \cite{Winner_2019}.

Nesse cenário de apropriação neoliberal das tecnologias digitais, o Núcleo de Tecnologia do Movimento dos Trabalhadores Sem Teto (MTST) emerge como uma experiência concreta de resistência ao colonialismo digital e à lógica extrativista das Big Techs. Enquanto as plataformas corporativas reforçam assimetrias de poder e dependência tecnológica – especialmente no Sul Global –, o núcleo propõe uma soberania digital popular, articulando tecnologia, educação crítica e organização política a partir das periferias urbanas.


\section{Pergunta de Pesquisa e Objetivos}
\label{sec:perg}
Para esta pesquisa, parto do interesse em compreender como práticas tecnológicas desenvolvidas em contextos populares podem constituir formas efetivas de resistência ao colonialismo digital e afirmação de soberania digital. Para isso, centro minha atenção nos artefatos desenvolvidos pelo Núcleo de Tecnologia do MTST, analisando sua concepção, funcionamento e uso como ferramenta tecnopolítica. A pergunta que guia este estudo é: de que maneira os artefatos tecnológicos desenvolvidos pelo Núcleo operam como ferramenta de resistência ao colonialismo digital e expressão de soberania digital popular, a partir de práticas tecnopolíticas e hacktivistas? Com essa investigação, pretendo refletir sobre como estratégias digitais produzidas por coletivos organizados podem tensionar as estruturas hegemônicas de poder nas plataformas digitais e contribuir para formas autônomas de organização social nas periferias do Sul Global.

\subsection{Objetivo Geral}
\label{subsec:objEs}

Analisar artefatos tecnológicos implementados pelo Núcleo de Tecnologia do MTST no contexto da luta por soberania digital popular e resistência ao colonialismo digital, investigando seus aspectos tecnológicos de implementação.

\subsection{Objetivos Específicos}
\label{subsec:objEs}

\begin{enumerate}
    \item Descrever a concepção, desenvolvimento e funcionamento técnico de algum artefato digital implementado pelo Núcleo.
    
    \item Analisar de que forma a atuação do Núcleo de Tecnologia do MTST, por meio deste artefato, expressa práticas de resistência ao colonialismo digital, à luz das dinâmicas do capitalismo de vigilância.

    \item Investigar como os conceitos de soberania digital são mobilizados pelo Núcleo de Tecnologia na construção e uso da ferramenta tecnológica.
    
\end{enumerate}

\section{Justificativa da Pesquisa}
\label{sec:just}

O projeto de pesquisa proposto se insere no cerne das disputas contemporâneas por soberanias digitais e busca compreender as formas de resistência e luta contra-hegemônica que emergem em um cenário de colonialismo digital e capitalismo de vigilância. Ao focar no Núcleo de Tecnologia do MTST (Movimento dos Trabalhadores Sem Teto), o estudo se dedica a analisar as dinâmicas digitais a partir de um saber localizado, como proposto por Donna Haraway e discutido por \citeonline{evangelista2017}.

\citeonline{evangelista2017} argumenta que os efeitos das tecnologias de vigilância e do Big Data não devem ser pensados como categorias universais, mas sim a partir das diferentes posições sociais. Ele fala como pesquisador de um país periférico ao sul do globo, preocupado com o destino de trabalhadoras e trabalhadores que vivem um cotidiano de crescente instabilidade econômica. Estudar o MTST permite analisar o capitalismo de vigilância e o colonialismo digital a partir da perspectiva e da experiência de um grupo socialmente vulnerável em um país periférico, o que, segundo o autor, nos obriga a pensar em especificidades e ênfases diferentes em relação à literatura vinda dos países centrais.

A importância do estudo também se fundamenta na crítica de \citeonline{Faustino2023} sobre o colonialismo digital, sendo o MTST um movimento social que já debate o tópico, buscando caminhos para a tecnopolítica e a descolonização da tecnologia. Além disso, o projeto aborda diretamente a problemática levantada pelos autores sobre o "abismo teórico que perpassa os dois polos" \cite[p.~30]{Faustino2023} entre a formação técnica e a humanística. Eles criticam a falta de compreensão da dimensão humana na produção tecnológica nos cursos técnicos e a superficialidade com que as ciências humanas tratam o funcionamento e atuação das tecnologias digitais. 

Estudar o Núcleo de Tecnologia do MTST oferece a oportunidade de compor essa interface de comunicação necessária ao analisar como um movimento social\footnote{Estou usando a denominação que eles empregam no site, mas trago o debate na próxima seção}, enraizado em lutas humanísticas e sociais, se apropria e se relaciona com as tecnologias digitais de forma crítica e estratégica. Isso permite ir além da mera denúncia do "fetiche da tecnologia" \footnote{Esses conceitos estão no livro de \cite{Faustino2023}, não sei se vou manter aqui na justificativa mesmo, se vou aprofundar eventualmente na revisão teórica} – a crença na neutralidade da técnica – e investigar as possibilidades de atuação política contra-hegemônicas, buscando "reintroduzir a política e a economia nesse debate sobre tecnologia" \footnote{Para retomar a discussão na revisão de literatura: \citeauthor{fisher_2020} levanta a questão de que  que nada é inerentemente politico. A politização requer um agente político que possa transformar o que é tido como garantido em algo a ser disputado. Essa questão é abordada no Cap. 2 do \cite{Silveira2021} também, checar fichamentos}.