% METODOLOGIA------------------------------------------------------------------

\chapter{METODOLOGIA}
\label{chap:metodologia}

Neste capítulo, será apresentada a abordagem metodológica que empregarei para investigar iniciativas do Núcleo de Tecnologia do MTST, como estratégia para a conquista da soberania digital popular e a promoção da resistência popular frente ao capitalismo de vigilância. A escolha metodológica reflete a necessidade de utilizar ferramentas apropriadas para analisar as complexas dinâmicas sociais, culturais e políticas que constituem o objeto deste estudo, bem como descrever os aspectos técnicos do desenvolvimento da ferramenta, analisar sua articulação com práticas tecnopolíticas e refletir sobre suas potencialidades.


\section{Abordagem e Natureza da Pesquisa}
A pesquisa terá uma abordagem qualitativa de natureza exploratória, adequada para a análise aprofundada de fenômenos sociotécnicos ainda pouco sistematizados na literatura. A investigação qualitativa permitirá interpretar sentidos e significados atribuídos às práticas e escolhas tecnológicas do Núcleo de Tecnologia, assim como os desafios enfrentados na construção de uma tecnologia popular, automatizada e autogestionada.
Conforme apontam \apudonline{appolinario2004}{menezes2019} e \apudonline{golsalves2003}{menezes2019}, a pesquisa qualitativa lidará com fenômenos e permitirá a análise hermenêutica\footnote{Segundo \citeonline{george2020}, hermanêutica é o estudo da interpretação, é aplicada quando objeto de estudo exige abordagens interpretativas, principalmente porque diz respeito ao significado das intenções, crenças e ações humanas, ou ao significado da experiência humana. No contexto deste trabalho, a hermenêutica seria descrita como um estudo "auxiliar" dos métodos e fundamentos da pesquisa, servindo como ferramenta de reflexão crítica sobre os caminhos metodológicos adotados.} dos dados coletados, considerando o significado que os integrantes do Núcleo atribuem às suas práticas tecnopolíticas de resistência, autogestão e luta por soberania digital.


O caráter exploratório se justifica pela necessidade de compreender em profundidade um fenômeno emergente — a interseção entre tecnologia e movimentos sociais — e pelo objetivo de iniciar a formulação e o teste de uma estrutura analítica preliminar que permita interpretar práticas tecnopolíticas em contextos de resistência. \apudonline{golsalves2003}{menezes2019} descreve a pesquisa exploratória como aquela que visa desenvolver e esclarecer ideias, oferecendo uma visão inicial e abrangente de fenômenos emergentes que ainda carecem de sistematização teórica.

\section{Tipo de Pesquisa}

O estudo configurará-se como um estudo de caso, concentrando a análise em uma iniciativa específica: o Núcleo de Tecnologia do MTST e algum artefato implementado por eles.

Adicionalmente, a pesquisa será orientada pelos princípios da pesquisa-ação, o que me permitirá uma interação colaborativa e dialógica com o Núcleo. \apudonline{thiollent1986}{menezes2019} e \apudonline{vergara2006}{menezes2019} definem a pesquisa-ação como um tipo de investigação que envolve a participação ativa dos sujeitos na resolução ou esclarecimento de problemas por meio de ações diretas, visando ampliar o conhecimento dos pesquisadores e a conscientização dos grupos envolvidos.

\section{Procedimentos e Técnicas de Coleta de Dados}

A pesquisa será delineada para combinar diferentes procedimentos e técnicas de coleta de dados, com o objetivo de obter uma compreensão abrangente do objeto de estudo. As técnicas de coleta de dados que utilizarei incluirão:

\begin{enumerate}

    \item \textbf{Revisão de Literatura:} Levantamento de textos sobre colonialismo digital, soberania digital, hacktivismo, capitalismo de vigilância e de plataformas e movimentos sociais urbanos.

    \item \textbf{Análise Documental:} Consulta a documentos produzidos pelo MTST e pelo Núcleo de Tecnologia, como o site oficial do MTST, o site do Núcleo de Tecnologia, documentos institucionais (como a cartilha \textit{"O MTST e a luta pela soberania digital a partir dos movimentos sociais"}), reportagens jornalísticas, entrevistas publicadas, transmissões online e artigos da própria organização. 

    \item \textbf{Análise de redes sociais e interações digitais:} Revisão de entrevistas já gravadas com membros do Núcleo de Tecnologia disponíveis em canais oficiais do MTST, documentários, ou veículos de mídia independente (ex.: transmissões ao vivo no YouTube, podcasts, reportagens), em especial os episódios gravados para o podcast Tecnopolíticas do professor Sérgio Amadeu da Silveira. Essa abordagem permitirá capturar perspectivas institucionais e individuais sem depender exclusivamente do acesso direto aos participantes.

    \item \textbf{Análise de redes sociais e interações digitais:} Mapear publicações, comentários e interações nas plataformas digitais utilizadas pelo Núcleo. A análise focará em padrões de comunicação entre o Núcleo e a base do movimento, bem como em estratégias de engajamento tecnopolítico.
    
    \item \textbf{Entrevistas:} Realizarei entrevistas\footnote{A realização de entrevistas com membros do Núcleo de Tecnologia está condicionada à obtenção de consentimento e acesso aos participantes. Caso o contato direto não seja viável, a análise priorizará documentos institucionais, registros públicos e entrevistas já disponíveis, garantindo rigor metodológico através da triangulação de fontes} para coletar dados primários e aprofundar a compreensão das experiências vividas pelos membros do Núcleo de Tecnologia e pessoas envolvidas no desenvolvimento e uso da ferramenta. 
\end{enumerate}

\section{Procedimentos para Realização da Pesquisa}

Os procedimentos de condução da pesquisa seguirão recomendações de manuais de metodologia científica. Embora este projeto ainda esteja em fase de elaboração, descrevo a seguir os procedimentos planejados.

A condução da pesquisa seguirá os princípios da triangulação metodológica, combinando dados da revisão bibliográfica, análise documental e entrevistas. Serão analisadas tanto a implementação do artefato tecnológico desenvolvido pelo Núcleo, quanto as interpretações dos sujeitos sobre seu papel nas lutas por soberania digital e resistência ao colonialismo digital. A análise priorizará o vínculo entre decisões técnicas e horizontes políticos, considerando os desafios da autogestão e das limitações de infraestrutura em contextos de periferia.

\section{Técnicas de Análise de Dados}

A análise dos dados coletados será conduzida sob uma perspectiva qualitativa. Utilizarei as seguintes técnicas:

\begin{itemize}
    \item \textbf{Análise Crítica do Discurso (ACD):} Aplicarei a ACD inspirada em \apudonline{fairclough2001}{salvagni2024}, conforme empregada por \citeonline{salvagni2024} para decodificar narrativas do Núcleo sobre soberania digital e tecnopolítica, contrastando-as com práticas empíricas. Utilizarei categorias como 'autonomia tecnológica' vs. 'dependência infraestrutural', inspiradas no modelo de \citeonline{salvagni2024}, que articulam economia solidária e ativismo digital 

    \item \textbf{Análise Temática/Categorial:} Submeterei os dados a um processo de codificação e organização em unidades de sentido, agrupadas em categorias. Essas categorias emergirão da análise do material empírico e poderão ser orientadas por estruturas analíticas já identificadas, como as propostas por \citeonline{salvagni2024}, que envolvem economia digital solidária, práxis do movimento social e política de base. A análise dessas categorias me permitirá discutir as estratégias de ativismo digital do MTST e contribuir para a construção de uma estrutura analítica.
\end{itemize}

\section{Considerações}

Considerando que a pesquisa envolverá seres humanos (entrevistas com membros do MTST e trabalhadores), seguirei rigorosamente os princípios éticos estabelecidos para esse tipo de pesquisa. O anonimato dos participantes será garantido mediante solicitação. Contudo, reconhece-se que a indisponibilidade de entrevistas diretas poderá limitar a profundidade da análise subjetiva, sendo compensada pelo cruzamento de fontes documentais e observacionais .


