% REVISÃO DE LITERATURA--------------------------------------------------------

\chapter{REVISÃO DE LITERATURA}
\label{chap:revisaodeliteratura}

\section{Colonialismo Digital}
\label{sec:colDados}

Segundo \citeonline{quijano2005}, o conceito tradicional de colonialismo, entendido como dominação política direta de um território por uma potência externa, é insuficiente para capturar a complexidade e a durabilidade dos padrões de poder estabelecidos a partir da conquista da América. A  colonização das Américas dá origem à ideia de raça como conhecemos hoje, as relações sociais que se fundamentam nessa base produz na América novas identidades, bem como redefine identidades já existentes. É nesse contexto que nascem o índio, o negro e o mestiço — e que renascem o  português, o espanhol e o europeu. 

Para além da procedência geográfica, esses termos agora adquirem uma conotação racial e são associados a identidades hierarquizadas dentro de um contexto de dominação, identidades essas que passam a ser associadas à natureza dos papéis na estrutura global de controle do trabalho. Dessa forma, é imposta uma sistemática divisão racial do trabalho: formas de trabalho forçado e não remunerado foram adscritas aos não-brancos nas colônias, enquanto o trabalho livre assalariado ficou limitado à raça colonizadora, brancos europeus \cite{quijano2005}. 

Todas formas de controle do trabalho na América foram deliberadamente estabelecidas e organizadas em torno e em função do capital e é essa realidade que fornece a base material para a acumulação primitiva europeia através do saque nas colonias, o que constitui então o capitalismo mundial e a Europa no centro desso novo sistema. Este processo estabelece um controle eurocêntrico dos mercados mundiais, fazendo da exploração colonial uma condição de existência do capitalismo \cite{quijano2005}.

Esta perspectiva nos ajuda a consolidar um entendimento sobre a continuidade dos padrões de dominação na era pós-colonial, pois enquanto o colonialismo político tem duração limitada no tempo e termina com a independência formal, a colonialidade perdura após a independência política, reproduzindo-se estruturalmente nas sociedades contemporâneas. É neste padrão de poder mundial que a hegemonia de instituições específicas moldam diversos âmbitos da existência social: a empresa capitalista controla o trabalho, a família burguesa articula as relações de sexo/gênero, o Estado-nação define a autoridade, e, por fim, o eurocentrismo influencia a intersubjetividade
 \cite{quijano2005}.

A colonialidade do poder é, portanto, fundamental para explicar a elaboração e hegemonia dessa perspectiva eurocêntrica sobre a subjetividade do colonizado\footnote{
Duas das características do colonialismo digital remontam mais diretamente ao colonialismo histórico, por se tratar de um controle material sobre recursos/territórios colonizados (controle da infraestrutura, controle da extração de minérios), mas, quando falo de colonialismo de dados e, mesmo, de capitalismo de vigilância, acho que é importante fundamentar a perspectiva de controle de subjetividades pelo eurocentrismo, por isso o conceito de colonialidade do poder casa bem aqui. Acho que vou retomar essa discussão ainda.
}.


\citeonline{Faustino2023} definem o colonialismo digital como a dinâmica do capitalismo tardio que constitui sua existência a partir de dois elementos intercambiáveis: uma nova repartição do mundo em espaços de exploração econômica e o colonialismo de dados.

Colonialismo de dados é definido como uma nova forma de colonialismo que se apoia na apropriação massiva da vida humana por meio da coleta contínua e exploração dos dados pessoais. Essa dinâmica combina práticas predatórias históricas do colonialismo tradicional com as capacidades computacionais atuais para quantificar comportamentos humanos. Trata-se não apenas de metáfora: é uma estrutura real que recria relações desiguais entre países ricos (Norte Global) e pobres ou em desenvolvimento (Sul Global), onde os fluxos de dados são extraídos predominantemente das periferias em direção às corporações tecnológicas concentradas no Norte. \cite{Silveira2021}

ETC.

\begin{quote}
Por colonialidade entende-se, conforme definição de Ballestrin, os mecanismos de dominação que se mantêm em funcionamento pelos países ricos mesmo após as independências. \citeauthor{Silveira2021}
\end{quote}

\subsection{Capitalismo de Vigilância}
\label{subsec:capVig}

\citeonline{Zuboff2019} define o Capitalismo de Vigilância\footnote{\label{nota:novoSistSoc}Segundo \citeauthor{Faustino2023} expressões como \nameref{subsec:capVig}, \nameref{subsec:capPlat}, etc sugerem a existência de um novo tipo de sistema social, distinto. Ignora ou secundariza alguns conceitos que auxiliariam na análise histórica do problema, o que fragiliza o debate e a percepção do que permanece do período anterior, embora agora intensificado ou reconfigurado com as novas possibilidades tecnológicas. } como um modelo de negócios em que a experiência humana é traduzida em dados comportamentais para prever e direcionar ações futuras, o direito à privacidade é usurpado por reivindicações unilaterais do mercado. As plataformas digitais se apropriam da experiência do usuário como matéria-prima gratuita, extraindo dados de suas interações, preferências e comportamentos. Neste modelo, o comportamento humano é arrebanhado em busca de resultados lucrativos e os meios de produção são subordinados aos "meios de modificação comportamental", os clientes deixam de ser os seres humanos, que viram objetos de extração de matéria prima, e se tornam as próprias empresas \cite{Zuboff2019}.
ETC.
\subsection{Capitalismo de Plataformas}
\label{subsec:capPlat}

Capitalismo de Plataformas\footnotemark[\value{footnote}] é definido como um capitalismo de dados tratados por algoritmos. 

Com a redução dos custos de computação e armazenamento de dados, aliada à consolidação da comunicação digital, Srnicek aponta a abertura de um novo potencial para o desenvolvimento de produtos e serviços orientados à extração de informações. Esses dados passaram a ser utilizados em processos produtivos, na identificação das preferências dos consumidores e no controle tanto dos trabalhadores quanto das cadeias logísticas \cite{silveira-demcodigos}. 

O armazenamento massivo de dados permite que desenvolvedores de algoritmos infiram categorias identitárias a partir dos hábitos de navegação dos usuários na internet, especialmente quando essas informações são cruzadas com outras bases de dados \cite{silveira-demcodigos}\footnote{Isso é na verdade uma paráfrase da frase original, que está na página 25: 
"O armazenamento de grande quantidade de dados oferece aos desenvolvedores de algoritmos a condição de inferir categorias identitárias baseadas nos hábitos de navegação na internet e no cruzamento com outras informações.". Achei interessante colocar aqui, mas não sei ainda se vou usar. Na minha visão, não são exatamente os desenvolvedores que inferem categorias identitárias — são os algoritmos, programados por eles, que realizam inferências com base nos dados. Embora os algoritmos sejam programados por desenvolvedores, esses profissionais não possuem acesso direto aos dados utilizados nem controle sobre os processos de inferência automatizada. Assim, a extração de perfis identitários é realizada por sistemas opacos cuja lógica de funcionamento escapa, em muitos casos, até mesmo àqueles que os constroem. Nesse mesmo livro o autor discute a questão da opacidade algorítmica: no primeiro capítulo, ao mencionar Lucas Introna e a ideia de que, mesmo com acesso ao código-fonte, é impossível compreender plenamente o funcionamento dos algoritmos. Posteriormente, no Cap. 4, aborda como as Big Techs intensificam essa opacidade ao manterem seus códigos proprietários fechados, dificultando ainda mais a transparência e a auditoria pública.}.

\section{Resistência}
\label{sec:resis}

\subsection{Soberania Digital}
\label{subsec:soberania}

\section{Núcleo de Tecnologia do MTST}
\label{sec:nucleomtst}

\subsection{O Movimento dos Trabalhadores Sem Teto }
\label{subsec:mtst}

Segundo informações disponíveis no próprio site da instituição, o Movimento dos Trabalhadores Sem Teto (MTST) é um movimento social urbano fundado em 1997 com o objetivo inicial de garantir o direito constitucional à moradia digna para todas as pessoas, estando presente em 14 estados brasileiros \cite{mtst_2024}. Sua principal forma de luta são as ocupações de terras que não cumprem função social nas periferias dos grandes centros urbanos \cite{mtst2023cartilha}.

De acordo com cartilha publicada no site do Núcleo de Tecnologia do MTST, o movimento teve origem a partir das experiências do Movimento dos Trabalhadores Rurais Sem Terra (MST), adaptando sua atuação às demandas do contexto urbano. No entanto, essa referência às origens não é mencionada no site oficial do MTST. Conforme a tese de doutorado da pesquisadora Débora Goulart, esse vínculo histórico também estava presente em uma outra cartilha, intitulada Cartilha do Militante, publicada em 2005 e atualmente indisponível online. Nessa publicação, o MTST reconhecia o MST como coautor de sua formação \cite{goulart2011anticapitalismo}. Apesar das diferentes interpretações sobre a origem do movimento, militantes do MTST confirmam o vínculo com o MST na sua criação, e é a partir de 2003 que o MTST se forma como um movimento autônomo em relação ao MST, com instâncias próprias  \cite{goulart2011anticapitalismo}.

Em sua tese, \citeauthor{goulart2011anticapitalismo} ressalta que o MTST se autorrefere como um movimento popular cujo elemento central é a classe, sendo uma organização dos trabalhadores. Em 2009, ele negava a alcunha de "movimento social" no sentido dado pela sociologia que, segundo eles, abarca praticamente qualquer tipo de mobilização, o que poderia esvaziar o caráter político e estruturante das lutas populares organizadas por trabalhadores. No entanto, em 2025, observa-se uma mudança de posicionamento: o próprio site do MTST refere-se ao grupo como um "movimento social" \cite{mtst_2024}. 

O movimento se considera legalista \cite{mtst2023cartilha}, ao fundamentar sua atuação na exigência de que os imóveis cumpram a função social prevista na Constituição Federal. Nos termos dos artigos 6º e 23, a moradia é reconhecida como um direito social e uma obrigação do Estado \cite{constituicao1988}. O projeto político do MTST é anticapitalista e antineoliberal. Ele busca a contestação do capitalismo por meio de lutas de ação direta, e entende a sociedade brasileira como capitalista onde opera a luta de classes. O movimento posiciona-se em oposição à exploração e dominação capitalista \cite{goulart2011anticapitalismo}.

Entre seus objetivos e atuações, além da moradia, destacam-se o combate à fome através do projeto Cozinhas Solidárias, que oferece alimentação diária e gratuita, utilizando recursos de hortas urbanas e doações. Eles oferecem diversas iniciativas de capacitação e assistência, como oficinas de construção de cisternas, cursos comunitários, oficinas de foto e vídeo, saraus e mutirões de assistência médica e jurídica \cite{mtst2023cartilha}.

A estrutura organizacional do MTST é composta por coletivos políticos e coletivos setoriais. As decisões estratégicas são formuladas nos coletivos políticos, enquanto os setoriais são responsáveis por implementá-las em suas respectivas áreas temáticas. Atualmente, o movimento conta com 13 setores, entre eles: Arquitetura, Arte e Cultura, Comunicação, Educação, Horta e Segurança Alimentar, Jurídico, Finanças e Captação de Recursos, Formação Política, Autodefesa, Organização, Negociação, Saúde e Assistência Social, além do Núcleo de Tecnologia, foco deste estudo \cite{mtst2023cartilha}.


\subsection{O Núcleo de Tecnologia}
\label{subsec:nucleotec}

A origem do núcleo se deu durante o momento de forte tensões políticas marcadas pelas eleicões presidencias de 2018 e esteve ligada à percepção, por parte de profissionais de tecnologia na base do movimento, de que o bolsonarismo tinha um enraizamento social real, e não se resumia apenas a presença de robôs nas redes sociais \footnote{Contextualizar o momento histórico aqui, com a onda "robôs" do bolsonaro. A cartilha dá a entender que nem todos enxergavam a atuação dos bolsonaristas nas redes como um movimento organizado, subestimavam a ameaça direta à democracia. }. Essa percepção catalisou debates internos sobre o papel estratégico da tecnologia na organização política, culminando na formalização do núcleo como um coletivo setorial dedicado a iniciativas tecnológicas \cite{mtst2023cartilha}.

Um primeiro passo foi a criação de um curso para desenvolvedores. Somado a isso, surgiram solicitações internas para a automação de tarefas cotidianas, como a conexão entre simpatizantes do movimento e trabalhadores da base que oferecem serviços sob demanda. Isso levou à criação do Contrate Quem Luta (CQL), um chatbot de WhatsApp que automatiza essa conexão. A escolha do WhatsApp como plataforma visou superar obstáculos enfrentados pelos trabalhadores da base, como smartphones com pouco espaço de armazenamento e acesso limitado à internet, aproveitando a gratuidade de tráfego de dados (zero rating) oferecida pelas empresas de telefonia \footnote{A justificativa é super valida, mas linkar aqui talvez com as problemáticas levantadas em \cite{Silveira2021} e \cite{Faustino2023}, WhatsApp é da Meta que tem a iniciativa Internet.org, que é basicamente uma estratégia de expansão do colonialismo digital, para colonizar a infraestrutura e o acesso digital, capturar dados >massivamente< e transformar a vida humana em ativos comercializáveis (mineração de dados, criação de perfis, etc.) e, assim, gerar lucro e exercer controle, tudo sob o pretexto de oferecer conectividade e conveniência (tem algo que é meio que institucionalizado, "plano de governo" mesmo, em cartilha formalizada pelo MIT(?) e seguida por governos mundo a fora, no sentido de transformação digital, checar nos fichamentos.}.


[TODAS AS INFOS ABAIXO SÃO APENAS DA CARTILHA, REVISAR]

O Núcleo também atua no combate à desinformação no ambiente digital e contribuiu para a elaboração de documentos importantes como o Plano de Ação para o Cooperativismo de Plataformas no Brasil e o Programa de Emergência para a Soberania Digital \cite{mtst2023cartilha}. [Pesquisar + Info sobre os planos]

No que tange a soberania digital, a posição do Núcleo é prágmática:
\begin{quote}
“O Núcleo de Tecnologia do MTST reivindica uma soberania digital que seja realmente pautada no fortalecimento da luta pelo poder popular na era da sociedade da informação. Queremos não só acesso significativo às tecnologias, à Internet, à educação digital e midiática, mas também direcionar o rumo tecnológico para quem verdadeiramente realiza a transformação social nos territórios.” \cite[p.~5]{mtst2023cartilha}
\end{quote}
O núcleo afirma que a técnica não é neutra e serve aos interesses de quem a constrói.
Eles reivindicam não apenas acesso à tecnologia e educação digital, mas também a capacidade de direcionar o rumo tecnológico para quem realiza a transformação social nos territórios. Consideram fundamental construir e manter ferramentas que atendam às necessidades do povo brasileiro sem dependência de agentes estrangeiros e suas regras, tornando a soberania digital um ponto central na construção da cidadania \cite{mtst2023cartilha}.

O Núcleo de Tecnologia ensina programação para jovens, adultos e crianças das ocupações e promove debates com pais e cuidadores sobre o risco da manipulação e exposição a conteúdos danosos em plataformas digitais. Buscam democratizar o acesso às formas de produzir e usar tecnologias de informação e comunicação para que a sociedade civil possa contornar estruturas políticas desfavoráveis, mobilizar recursos, fazer campanhas, engajar a militância e documentar a memória das lutas. Sua atuação inclui participação em debates no circuito tecnológico brasileiro e parcerias com outros movimentos sociais.

A estrutura e organização do Núcleo são vistas como práticas e sua composição é descrita em camadas: na parte mais externa estão os simpatizantes, sem qualquer atuação direta ou indireta. A próxima camada são os apoiadores, que sentem a necessidade de ajudar de alguma forma (doação, divulgação, etc.). Em seguida, vêm os ativistas, que são frequentes nas atividades e atuam em momentos como mutirões e manifestações. Por fim, estão os militantes, contribuindo no cotidiano do movimento. O Núcleo lida com um engajamento flutuante e por isso busca harmonizar esses diferentes perfis para manter a regularidade.

\section{Trabalhos relacionados}
\label{sec:trabalhosrelacionados}

