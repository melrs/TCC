% RESUMO--------------------------------------------------------------------------------

\begin{resumo}[RESUMO]
\begin{SingleSpacing}

% Não altere esta seção do texto--------------------------------------------------------
\imprimirautorcitacao. \imprimirtitulo. \imprimirdata. \pageref {LastPage} f. \imprimirprojeto\ – \imprimirprograma, \imprimirinstituicao. \imprimirlocal, \imprimirdata.\\
%---------------------------------------------------------------------------------------

A crescente centralidade das tecnologias digitais e das plataformas de dados no cotidiano das sociedades contemporâneas tem aprofundado assimetrias de poder, especialmente no Sul Global, onde a extração de dados e a dependência de infraestruturas controladas por grandes corporações tecnológicas configuram novas formas de colonialismo digital e capitalismo de vigilância. Esta projeto de Trabalho de Conclusão de Curso propõe uma investigação de como o Núcleo de Tecnologia do Movimento dos Trabalhadores Sem Teto (MTST) atua na resistência a essas dinâmicas, como experiência concreta de tecnopolítica popular e busca por soberania digital.

A pesquisa adotará uma abordagem qualitativa e exploratória, fundamentada em revisão de literatura, análise documental, observação de interações digitais e entrevistas (diretas ou já gravadas) com membros do Núcleo e usuários do artefato. Utilizará a análise crítica do discurso e categorização temática para examinar tanto os aspectos técnicos quanto as dimensões políticas e éticas da iniciativa. Os resultados esperados incluem a identificação de estratégias de resistência digital, a problematização das contradições inerentes ao uso de plataformas hegemônicas por movimentos sociais e a proposição de caminhos para a construção de alternativas tecnológicas autônomas e decoloniais.

O estudo contribui para o debate sobre soberania digital popular, evidenciando os desafios e potencialidades de práticas tecnopolíticas em contextos periféricos e propondo uma reflexão crítica sobre as possibilidades de reapropriação e transformação das tecnologias digitais a partir das lutas sociais.

\textbf{Palavras-chave}: Colonialismo Digital. Soberania Digital. Capitalismo de Vigilância. MTST. Tecnopolítica. Resistência Popular.
% Escolha de 3 a 5 palavras ou termos que descrevam bem o seu trabalho 
\end{SingleSpacing}
\end{resumo}


